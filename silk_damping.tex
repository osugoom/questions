\documentclass{article}
\usepackage{amsmath}
\usepackage[margin=0.75in]{geometry}
\usepackage{graphicx}

\def\beq{\begin{eqnarray}}
\def\eeq{\end{eqnarray}}

\title{Silk Damping}
\author{Andr\'{e}s N. Salcedo}

\begin{document}

\maketitle

Silk damping refers to the erasure of small scale perturbations in the (baryonic) matter-photon field at early times. The matter is coupled to the photons and so is dragged along during diffusion from hot/dense to cold regions. You may use the following equation relating scale factor and time in a universe with radiation and matter long after matter-radiation equality:

$$ a \approx \left( \frac{3}{2} \sqrt{\Omega_m} H_0 t \right)^{\frac{2}{3}} $$

\begin{enumerate}

\item Calculate the mean free path for Silk damping photons.

\item What is the scale of density perturbations that will be washed out by Silk damping?

\item What is the mass contained in a region of this scale? What implications does this have for observed structure at $z=0$?
  
\end{enumerate}

\end{document}
