\documentclass[a4paper]{article}
\usepackage[utf8]{inputenc}
\usepackage{siunitx}
\title{Interstellar Propulsion}
\author{Matthias Raives}
\date{\today}
\begin{document}
  
  \maketitle
  
  \textbf{(a)} The Bussard Ramjet is a hypothetical means of space travel in which a spaceship travelling through the ISM uses magnetic fields to capture protons and feed them into a fusion reactor for propulsion.  What is the maximum speed of a spacecraft equipped with the Bussard Ramjet, assuming perfect efficiency of the reactor (i.e., all produced energy is used for propulsion and the excess protons produced by the p-p chain are recaptured losslessly), and that the reaction is fuel-limited?
  
  \textbf{(b)} Project Orion was a hypothetical spacecraft design that used nuclear pulse propulsion -- nuclear bombs would be exploded behind the spacecraft, propelling it forward.  What fuel-to-payload mass ratio do you need to reach the same velocity as the Bussard Ramjet, assuming U-235 fission bombs?\\
  \\
  Proton-Proton chain:
  \begin{equation}
      6\mathrm{p} \rightarrow \mathrm{He}^{4} + 2\mathrm{p} + \nu_{e} + \SI{7}{\mega\eV}
  \end{equation}
  \noindent U-235 fission:
  \begin{equation}
      {}^{235}\mathrm{U} + \mathrm{n} \rightarrow {}^{141}\mathrm{Ba} + {}^{92}\mathrm{Kr} + 3\mathrm{n} + \SI{200}{\mega\eV}
  \end{equation}
  
  \noindent Energy yields are approximate.
  
\end{document}
