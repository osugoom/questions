\documentclass[a4paper]{article}
\usepackage[utf8]{inputenc}
\usepackage{amsmath,amssymb,siunitx,cancel}

\def\mhe{m_{\mathrm{He}}}

\title{Interstellar Propulsion}
\author{Matthias Raives}
\date{\today}
\begin{document}
 
  \maketitle
  
  \section{Bussard Ramjet}
    This can be derived multiple ways both relativistic and non-relativistic.  The terminal velocity is large enough that the relativistic derivation is preferred, but the non-relativistic answers are correct to an order-of-magnitude level (within a factor of 3).  Three derivations that I have found are presented here.
    
    \subsection{Relativistic Monentum Conservation}
      The spacecraft generates thrust from the proton-proton chain:
      \begin{equation}
          6\mathrm{p} \rightarrow \mathrm{He}^{4} + 2\mathrm{p} + \nu_{e} + \Delta E
      \end{equation}
      Since our reactor is perfectly efficient, we'll assume that all the $\Delta E$ goes into launching the neutrino (which we'll assume is negligible) and helium nucleus out of the spaceship for propulsion, and that we capture the two extra protons produced (so that, in the steady state, we only need to capture 4 protons per reaction).
      
      At terminal velocity, the momentum of the ejected helium nuclei is equal to the total momentum of the protons we're collecting.  The momentum of the protons is:
      \begin{equation}
          p_{1} = \sum p_{p} = 4\gamma m_{p}v_{T} = \frac{4m_{p}v_{T}}{\sqrt{1-\frac{v_{T}^{2}}{c^{2}}}},
      \end{equation}
      and the momentum of the exhaust can be determined by:
      \begin{align}
          E = \mhe c^{2} + \Delta E &= (\mhe^{2}c^{4} + p_{2}^{2}c^{2})^{1/2}\\
          (\mhe+\Delta m)^{2}c^{4} &= \mhe^{2}c^{4} + p_{2}^{2}c^{2}\\
          \cancel{\Delta m^{2}} + \mhe\Delta m &= p_{2}^{2}c^{-2}\\
          p_{2} &= c\sqrt{\mhe\Delta m}.
      \end{align}
      The helium nucleus mass is about $\mhe\sim 4 m_{p}$, and the energy produced by the proton-proton chain converts to a mass:
      \begin{equation}
          \Delta m = \frac{\Delta E}{c^{2}} \sim 10^{-2}m_{p}
      \end{equation}
      Setting these momenta equal:
      \begin{align}
          p_{1} &= p_{2}\\
          \frac{4m_{p}v_{T}}{\sqrt{1-\frac{v_{T}^{2}}{c^{2}}}} &= c\sqrt{\mhe\Delta m}\\
          \frac{\frac{v_{T}}{c}}{\sqrt{1-\frac{v_{T}^{2}}{c^{2}}}} &= \frac{\sqrt{\mhe\Delta m}}{4m_{p}}\\
          \frac{v_{T}}{c} &= \left(\frac{1}{\frac{16m_{p}^{2}}{\mhe\Delta m}+1}\right)^{1/2}\\
          \frac{v_{T}}{c} &= \left(\frac{1}{400+1}\right)^{1/2} \sim \frac{1}{20} = 0.05
      \end{align}
    
    \subsection{Non-relativistic Momentum Conservation}
      This method is similar to the first, but it uses the non-relativistic formulas.  We have the total momentum of the protons:
      \begin{equation}
          p_{1} = \sum p_{p} = 4m_{p}v_{T},
      \end{equation}
      and the exhaust momentum:
      \begin{equation}
          p_{2} = \sqrt{2\mhe\Delta E} = c\sqrt{2\mhe\Delta m}.
      \end{equation}
      Setting these equal:
      \begin{align}
        p_{1} &= p_{2}\\
        4m_{p}v_{T} &= c\sqrt{2\mhe\Delta m}\\
        \frac{v_{T}}{c} &= \left(\frac{\mhe\Delta m}{8m_{p}^{2}}\right)^{1/2}\\
        \frac{v_{T}}{c} &= \frac{\sqrt{2}}{20} \sim 0.07
      \end{align}
    
    \subsection{Non-relativistic Drag}
      The generalized drag formula is:
      \begin{equation}
          \frac{1}{2}\rho v_{T}^{2}A = F
      \end{equation}
      where $A$ is the effective collecting area of our craft.  In this case, $F$ is the thrust generated by our reactor.  We can approximate $F$ by:
      \begin{equation}
          F = \frac{\mathrm{d}p}{\mathrm{d}t} \sim \frac{p}{4\tau} = \frac{pv_{T}}{4\lambda}
      \end{equation}
      where $\tau = \frac{\lambda}{v_{T}}$ is the mean time between interactions with ISM protons, and $\lambda$ is the mean free path of our spaceship in the ISM, given by
      \begin{equation}
          \lambda = \frac{m_{p}}{\rho A}.
      \end{equation}
      Thus, we have:
      \begin{equation}
          \frac{1}{2}\rho v_{T}^{2}A = \frac{p\rho Av_{T}}{4m_{p}}.
      \end{equation}
      The exhaust momentum $p$ is the same as in the earlier non-relativistic derivation.  Solving for $v_{T}$:
      \begin{equation}
          \frac{v_{T}}{c} = \frac{1}{2c}\frac{p}{m_{p}} = \left(\frac{\mhe\Delta m}{2m_{p}^{2}}\right)^{1/2} = \frac{\sqrt{2}}{20} \sim 0.14
      \end{equation}
    
  \section{Project Orion}
    \subsection{Rocket Equation}
    Integral to this question is the Rocket Equation:
    \begin{equation}
      \Delta v = v_{e}\ln\left(\frac{m_{0}}{m_{f}}\right)
    \end{equation}
    The derivation is as follows.  At time $t=0$, the rocket has mass $m+\Delta m$ and velocity $v$.  At time $t=\Delta t$, the rocket has velocity $v+\Delta v$, mass $m$, and the spent fuel has mass $\Delta m$ and velocity $v-v_{e}$, where $v_{e}$ is the rest frame ejecta velocity.  By conservation of momentum, then:
    \begin{align}
      (m+\Delta m)v &= m(v+\Delta v) + \Delta m (v-v_{e})\\
      mv + \Delta m v &= mv + m\Delta v + \Delta m v - \Delta mv_{e}\\
      \Delta m v_{e} = m\Delta v\\
      \Delta v &= v_{e}\frac{\Delta m}{m}
    \end{align}
    Integrating this yields:
    \begin{equation}
     \Delta v = v_{e}\ln\left(\frac{m_{0}}{m_{f}}\right)
    \end{equation}
    
    \subsection{Exhaust Velocity}
      In the case of pulse propulsion, the ``ejecta velocity'' is actually the velocity of the explosion, when it pushes against your ship.  A nuclear bomb of mass $M$ has yield $E = \epsilon Mc^{2}$; the ejecta then has velocity:
      \begin{equation}
        v_{e} = \sqrt{\frac{2E}{M}} = \sqrt{2\epsilon c^{2}}
      \end{equation}
      However, only a fraction $f\leq \frac{1}{2}$ of the mass is intercepted, so the effective velocity is
      \begin{equation}
       v_{e} = f\sqrt{2\epsilon} c
      \end{equation}
    
    \subsection{Mass Ratio}
      Adopting a fiducial $\epsilon=10^{-2}$ and $f=\frac{1}{2}$, we see:
      \begin{equation}
       \frac{m_{0}}{m_{f}} = \exp\left(\frac{\Delta v}{v_{e}}\right) = e
      \end{equation}
      That is, slightly over half the mass of the spacecraft must be fuel (i.e., nukes).  The answer here will vary considerably because of the exponential, however.
  
\end{document}
