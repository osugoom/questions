\documentclass[12pt, letterpaper]{article}
\usepackage{amsmath}
\usepackage{geometry}
 \geometry{
 bottom=20mm,
 right=20mm,
 left=20mm,
 top=20mm,
 }


\newcommand{\Kel}{\mathrm{K}}
\newcommand{\OmegaB}{\Omega_b}
\newcommand{\rhob}{\rho_b}
\newcommand{\rhocrit}{\rho_{\mathrm{crit}}}
\newcommand{\rhoIGM}{\rho_{\mathrm{IGM}}}
\newcommand{\nIGM}{n_{\mathrm{IGM}}}

\title{Decoupling}
\author{Andr\'{e}s N. Salcedo}
\date{December 1, 2017}

\begin{document}

\maketitle

Generically decoupling refers to a period when different types of particles fall out of thermal equilibrium with each other. Photon decoupling refers to the period in the development of the universe when the density of electrons dropped enough to allow photons to stream freely. These photons are now observed as part of the CMB.  Neutrinos similarly decoupled at an earlier redshift, and thus there should be an analagous cosmic neutrino background.

\begin{enumerate}

\item Calculate the redshift of photon decoupling.

\item (Bonus) Calculate the redshift of neutrino decoupling. For neutrino interactions with positrons and electrons $\sigma v \approx G^2_F T^2$ where $G_F = 1.16 \times 10^-5 \; \mathrm{GeV}^{-2}$ is Fermi's constant. You may also make use of the fact that number density of relativistics electrons and positrons can be written $n \propto T^3$.
 

\end{enumerate}

\end{document}
