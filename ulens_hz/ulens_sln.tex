\documentclass[12pt]{article}

\usepackage[bindingoffset=0.2in,left=0.5in,right=0.5in,top=1in,bottom=1in,footskip=.25in]{geometry}
\usepackage{graphicx}
\graphicspath{ {images/} }
\usepackage{hyperref}
\usepackage{amsmath}
\usepackage{amsfonts}
\usepackage{cite}
\usepackage{fancyhdr}
\usepackage{physics}

\newcommand \de[2]  { \frac{\mathrm d{#1}}{\mathrm d{#2}}   }

\pagestyle{fancy}

\rhead{Samson Johnson}
\chead{Habitable Zone Microlensing}
\lhead{GOOM}
\setlength{\headheight}{15pt}



\begin{document}
\raggedright
\begin{center}
\textit{``The plural of anecdote is data.''}\\
\hspace{20mm}\textit{-Raymond Wolfinger}
\end{center}

\begin{itemize}
\item[1.] Using microensing, planets are most easily discovered when they have a semi-major axis close to the Einstein ring radius of the star, 
\begin{equation}
r_E = \sqrt{2R_s\frac{D_dD_{ds}}{D_s}},
%r_E = \sqrt{\frac{4GM}{c^2}\frac{D_dD_{ds}}{D_s}},
\end{equation}
for the Schwarzschild radius $R_s$,  the distance to the source star $D_s$, the distance to the lens (deflector) $D_d$, and $D_{ds}=D_s-D_d$. 
%Let's assume the the approximate habitable zone of a star scales as the root of its luminosity, 
%\begin{equation}
%r_{HZ} = 1\text{AU}\left(\frac{L}{L_\odot}\right)^{1/2}.
%\end{equation}
At what stellar mass is the Einstein ring radius equal to the radius of the habitable zone for a lens distance of 0.1 Kpc and a source distance of 8 Kpc. 

\item[1.] We have an equation for the physical size of the Einstein ring radius, but we lack a relation for the semi-major axis of the habitable zone. A simple scaling would be to assume that it scales as the root of the luminosity of the star, or 
\begin{equation}
r_{HZ} = 1\text{AU}\left(\frac{L}{L_\odot}\right)^{1/2}.
\end{equation}
Now, we can use a mass luminosity relation, $L/L_\odot\approx (M/M_\odot)^4$. Let's now simplify our expression for $r_E$. Our lens is very close to us, so we will assume $D_s\approx D_{ds}$.
\begin{equation}
r_E = \sqrt{\frac{4GM}{c^2}D_d} \approx 1\textrm{AU} \left(\frac{M}{M_\odot}\right)^{1/2}= 1\textrm{AU}\left(\frac{M}{M_\odot}\right)^2.
\end{equation}
So the mass of the star has to be 1 $M_\odot$.


\item[2.] Let's say we monitor a large number of bulge stars, $N_*$. The microlensing event rate is 
\begin{equation}
\Gamma = \frac{2N_*\tau}{\pi t_e}.
\end{equation}
\begin{itemize}
\item[a.] The Einstein ring crossing time $t_E=\theta_E/\mu_{rel}$, where $\theta_E=r_E/D_d$ is the angular size of the Einstein ring and $\mu_{rel}$ is the relative proper motion between the source and the lens.
\begin{itemize}
\item[i.] For our lens at 0.1 Kpc, assume the source has neglibible proper motion ($\mu_{rel} = \mu_{lens}$). The lens has a a transeverse velocity of 10 km/s w.r.t. the Earth. Find the proper motion in mas/year (better) or rad/year (more useful).

\item[i.] Just math it up, should get 20 mas/yr.

\item[ii.] Determine $t_E$ for this event.
\item[ii.] $\theta_E$ should be 10 mas. This will give you $t_E=0.5$ yr. 
\end{itemize}
\item[b.] The microlensing optical depth $\tau$ is sort of like a lensing probability. We can think about it as the fraction of sky covered by the angular area of the lenses' E-rings,
\begin{equation}
\tau = \frac{1}{\Omega}\int_0^{D_s} \left[n(D_d)D_d^2(\pi\theta_E^2)dD_d\right].
\end{equation}
Find the `optical depth' for the lens star we have been considering so far. 
\item[b.] We can say that our lens distribution is just a Dirac delta function at the location of the lens, or 
\begin{equation}
\tau = \frac{1}{\Omega}\int_0^{D_s} \left[\delta(D_d-0.1\textrm{Kpc})D_d^2(\pi\theta_E^2)dD_d\right]=10^{-2}\frac{\pi\theta_E^2}{\Omega}.
\end{equation}
Let's assume we monitor $\Omega=9$ deg$^2$. This should result in $\tau\approx2\times10^{-14}$. A very low lensing probability indeed.
\item[c.] What is the event rate $\Gamma$ for our habitable zone planet if we watch $10^8$ potential source stars?
\item[c.] This should come out to be $3\times10^{-6}\textrm{yr}^{-1}$.
\end{itemize}
\end{itemize}

\end{document}
