\documentclass{article}

\usepackage{amsmath,amssymb,siunitx,graphicx}
\usepackage[margin=1in]{geometry}
\DeclareSIUnit\ergs{ergs}
\DeclareSIUnit\yr{yr}
\DeclareSIUnit\AU{AU}
\DeclareSIUnit\msun{\ensuremath{\mathrm{M}_{\odot}}}

\title{Snowfall}
\author{Matthias J. Raives}

\begin{document}
    
    \maketitle{}
    
    \begin{enumerate}
    
        \item How much snow (fractionally) is melted by the kinetic energy of a new layer of snowfall?  Answer both in terms of the mass and height of snow melted.  Assume the snow has equilibrated to an ambient temperature of \SI{0}{\celsius}.
        
        %~ \item Conversely, how much rainwater is evaporated by the kinetic energy of a new "layer" of rain?
        
        \item Consider now hail instead of snow.  How large does the hail need to be before it melts more ice than it deposits?
    
    \end{enumerate}    
    
    The specific heat of fusion of water is $\Delta\epsilon_\mathrm{fus}=\SI{3e9}{\ergs\per\gram}$.
    
\end{document}
