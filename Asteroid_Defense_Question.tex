\documentclass{article}

\usepackage{amsmath,amssymb,siunitx,graphicx}
\usepackage[margin=1in]{geometry}
\DeclareSIUnit\ergs{ergs}
\DeclareSIUnit\yr{yr}
\DeclareSIUnit\AU{AU}
\DeclareSIUnit\msun{\ensuremath{\mathrm{M}_{\odot}}}

\title{Asteroid Defense}
\author{Matthias J. Raives}

\begin{document}
    
    \maketitle{}
    
    The Chicxulub impactor was a $\sim\SI{10}{\km}$ diameter asteroid that struck the Yucat\'an peninsula 65 million years ago, knocking a large mass of dust into the atmosphere, leading to a chain reaction that wiped out most of the megafauna of the time, i.e., the dinosaurs.
    
    \begin{enumerate}
        
        \item Suppose a similarly-sized asteroid is detected at the oribital distance of the moon, heading directly for the center of the earth.  How much energy would be required to deflect the asteroid onto a path where it misses the earth (as measured in the reference frame initially co-moving with the asteroid)?
        
        \item Suppose that, instead of deflecting the asteroid, NASA decides to send a crack team of oil drillers to the asteroid to plant a bomb in the center, fragmenting it.  How many (equal-sized) fragments must the asteroid be broken up into in order to decrease the total impact energy by a factor of 10, assuming all fragments impact the earth?
        
    \end{enumerate}
    
\end{document}
